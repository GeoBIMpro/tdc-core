\documentclass[a4paper,11pt]{article}
\usepackage{fullpage}
\usepackage[latin1]{inputenc}
\usepackage[T1]{fontenc}
\usepackage[normalem]{ulem}
\usepackage[english]{babel}
\usepackage{listings,babel}
\lstset{breaklines=true,basicstyle=\ttfamily}
\usepackage{graphicx}
\usepackage{moreverb}
\usepackage{url}

\title{Time Domain Converter core for Spartan-6 FPGAs}
\author{S\'ebastien Bourdeauducq}
\date{August 2011}
\begin{document}
\setlength{\parindent}{0pt}
\setlength{\parskip}{5pt}
\maketitle{}
\section{Specifications}

\section{Calibration mechanism}
In the formulas below:
\begin{itemize}
\item $T_{sys}$ is the system clock period.
\item $H(n)$ is the number of hits in the histogram for bin $n$.
\item $W(n)$ is the width of bin $n$.
\item $C = \displaystyle\sum\limits_{n} H(n)$ is the total number of hits in the histogram.
\item $R(n)$ is the time stamp of an event whose signal propagated up to bin $n$. The LUT contains the function $R$.
\item $f$ (respectively $f_{0}$) is the current (respectively reference) frequency of the online calibration ring oscillator.
\end{itemize}

\subsection{Offline calibration}
\begin{equation}
W_{0}(n) = \frac{H(n)}{C} \cdot T_{sys}
\end{equation}

\begin{equation}
R_{0}(n) = \displaystyle\sum\limits_{i=0}^{n}{W_{0}(i)} = \frac{T_{sys}}{C} \cdot \displaystyle\sum\limits_{i=0}^{n}{H(i)}
\end{equation}

\subsection{Online calibration}

\begin{equation}
R(n) = \frac{f_{0}}{f} \cdot R_{0}(n)
\end{equation}

\end{document}
